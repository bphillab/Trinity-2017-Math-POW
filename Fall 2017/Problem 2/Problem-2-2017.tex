\documentclass[11pt]{article}
\usepackage{geometry}                % See geometry.pdf to learn the layout options. There are lots.
\usepackage{framed}
\usepackage{amsfonts}
\usepackage{amssymb}
\usepackage{mathtools}
\geometry{letterpaper}  
\newtheorem{theorem}{Theorem}[section]
\newtheorem{lemma}[theorem]{Lemma}
\newtheorem{proposition}[theorem]{Proposition}
\newtheorem{corollary}[theorem]{Corollary}

\newenvironment{proof}[1][Proof]{\begin{trivlist}
\item[\hskip \labelsep {\bfseries #1}]}{\end{trivlist}}
\newenvironment{definition}[1][Definition]{\begin{trivlist}
\item[\hskip \labelsep {\bfseries #1}]}{\end{trivlist}}
\newenvironment{example}[1][Example]{\begin{trivlist}
\item[\hskip \labelsep {\bfseries #1}]}{\end{trivlist}}
\newenvironment{remark}[1][Remark]{\begin{trivlist}
\item[\hskip \labelsep {\bfseries #1}]}{\end{trivlist}}

\newcommand{\qed}{\nobreak \ifvmode \relax \else
      \ifdim\lastskip<1.5em \hskip-\lastskip
      \hskip1.5em plus0em minus0.5em \fi \nobreak
      \vrule height0.75em width0.5em depth0.25em\fi}
                 % ... or a4paper or a5paper or ... 
%\geometry{landscape}                % Activate for for rotated page geometry
\usepackage[parfill]{parskip}    % Activate to begin paragraphs with an empty line rather than an indent
\usepackage{epstopdf}
\usepackage{amsmath}
\title{Math Problem Of The Week: Problem 2\\Week of September 5, 2017}
\author{Benjamin Phillabaum\\Northbrook, IL}
\begin{document}
\maketitle
\newpage

\begin{framed}
Assume that $x$, $y$, and $z$ are all positive, real numbers that satisfy the system of equations
\begin{eqnarray*}
x+y+xy&=&8\\
y+z+yz&=&15\\
x+z+xz&=&35
\end{eqnarray*}
Determine the value of $x+y+z+xyz$.
\end{framed}
I begin by noting that the above expressions can be simplified notationally to 
\begin{eqnarray*}
x_i + x_j +x_i x_j &=& \lambda_k\\
(x_i+1) (x_j+1) &=& \lambda_k+1
\end{eqnarray*}
for $i\neq j \neq k$. This allows me to deduce each of the $x_i$:
\begin{equation*}
x_i = \sqrt{\frac{(\lambda_k+1)(\lambda_j+1)}{\lambda_i + 1}}-1
\end{equation*}
Thus
\begin{eqnarray*}
x &=& \frac{7}{2}\\
y &=&1\\
z&=&7 \\
x+y+z+xyz &=& 36
\end{eqnarray*}

$\square$
\end{document}